\chapter*{Závěr}
\addcontentsline{toc}{chapter}{Závěr}
\vspace{-10mm}
Přínos tohoto výzkumného úkolu lze vidět zejména ve vylepšení DicomPresenteru. Z programu byly odstraněny další chyby, které bránily v jeho nasazení do provozu. V programu přibyla nová funkce: multiplanární rekonstrukce. Díky multiplanární rekonstrukci se program opět přiblížil profesionálním prohlížečům.

Co se týče teoretické části této práce: bylo zjištěno, že z programu lze vyjmout knihovny OpenGL a Cg toolkit, jež činili potíže s kompatibilitou. Byly nalezeny postupy, jak realizovat manipulace se snímky bez použití knihoven. Ve výzkumném úkolu jsme odhadli, jak rychle by měl běžet DicomPresenter bez jmenovaných knihoven. Dále byla objevena možnost, jak lépe počítat změnu kontrastu snímku pomocí nelineární transformace.

V příštím školním roce si autor klade za cíl odstranit z DicomPresenteru obě jmenované knihovny, což by mělo výrazně přispět k bezproblemovému chodu aplikace. Součástí toho bude výrazná refaktorizace zdrojového kódu. DicomPresenter by pak už mohl být použitelný v IKEM.