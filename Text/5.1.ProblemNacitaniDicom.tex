\section{Problém s načítáním DICOM snímků}
Situace na poli prohlížečů lékařských snímků je dosti komplikovaná. V práci\cite{flaska} na str. 10 je to blíže vysvětleno: Prohlížeče DICOM snímků jsou buď komerčními placenými produkty a nebo jsou volně šiřitelné, ale nedosahují takových kvalit. Při vývoji DicomPresenteru byl program testován na určité množině snímků (např. archiv ženevské university\citesec{DicomSamples}). Velice známý prohlížeč Medinria, pak nezanedbatelnou část těchto snímků nedokázal správně zobrazit (buď je nenačetl vůbec, nebo je zobrazil s chybami). Tento fakt jen dokazuje to, že volně šiřitelné prohlížeče jsou zatím zpravidla ve fázi vývoje a vývoj DicomPresenteru tak má smysl, neboť pomyslný trh zatím není žádným významným prohlížečem obsazen (jiná situace je u Mac OS X\footnote{Pro operační systém Mac OS X byl vytrořen velice úspešný prohlíčeč OsiriX. Z tohoto důvodu není moc vhodné DicomPresenter v budoucno portovat pro Mac OS X, neboť tam má prohlížeč silnou konkurenci a tak by o něj zřejmě málokdo jevil zájem.}).

S podobnými problémy s jakými se potýkají volně šiřitelné prohlížeče se však potýká i DicomPresenter. Při rozsáhlejším testování programu na větší množině dat nastal problém, že prohlížeč nedokázal načíst poměrně rozsáhlou skupinu DicomSnímků (jiná část snímků ze zmíněného archivu), navíc prohlížeč při pokusu o načtení takového snímku zhavaroval. To se jevilo jako dost závažný problém.

Při bližším zkoumání, proč snímky nelze nalézt, se podařilo objevit, v které části kódu nastává problém:

\begin{lstlisting}[label=DicomImageClass,caption={První část souboru \texttt{Window.cpp} se zdrojovým kódem třídy reprezentující okno programu.}]
bool CDicomFrames::LoadDicomImage(char *fileName, bool isFirst, int framesCount ) {
	iDicomImage = new DicomImage (fileName);
	...	
}
\end{lstlisting}

Přičemž je potřeba zdůraznit, že \clist{DicomImage} je třída knihovny \clist{dcmtk}. Její konstruktor je definován:
\begin{lstlisting}[label={DicomImage}]
DicomImage::DicomImage (const char *filename,
    const unsigned long flags = 0,
    const unsigned long fstart = 0,
    const unsigned long fcount = 0)
{
...
}
\end{lstlisting}

a dále hodnota předávaného parametru \clist{fileName} byla:

\clist{"C:/Dev/Samples/IMG-0001-0001.dcm"}.

Z uvedeného vyplívá, že kontruktor třídy byl volán správně a se správným parametrem. Problém tedy vznikal někde uvnitř knihovny \clist{dcmtk}, která nedokázala správně načíst požadovaný soubor. Zdůrazněme ještě, že se jednalo o DICOM soubory z archivu Ženevské universitní nemocnice určené pro studijní účely.

\subsection{Hledání příčiny selhání}
Jak bylo popsáno výše, knihovna DCMTK zhavarovala při načítání DICOM snímku, nebylo však možné zjistit v které části dcmtk knihovny nastal problém. Pro bližší diagnostiku problému v externí knihovně je potřeba si knihovnu přeložit v debug verzi, kdy si pak můžeme nechat vypsat konkrétní řádek na kterém nastala chyba (viz. str. \pageref{releasedebug}).

Problém byl však v tom, že pokud byla knihovna přeložena v Release konfiguraci, tak obrázek nedokázala načíst a při načítání obrázku program havaroval: došlo k neoprávněnému čtení z/zápisu do místa v paměti, které nebylo pro program vyhrazeno (access violation, respektive segmentation fault (viz. práce \cite{flaska} strana 39)). Pokud byla knihovna přeložena v Debug konfiguraci, načítání obrázku proběhlo do konce, program nezhavaroval a běh programu dále pokračoval. Příčina tohoto rozdílného chování je popsána na straně \pageref{releasedebug} této práce.

Běžná praxe u používaných C++ knihoven, které pracují s externími daty je taková, že si lze v průběhu programu vyžádat od knihovny, aby nám popsala, jak dopadl poslední požadavek:

\hspace{-1cm}
\begin{tabular}{| p{3cm} | p{3cm} | p{2cm} | p{6cm} | }
  \hline                       
  Knihovna & Funkce & Typ & Příklady \\
  \hline
  \hline                     
  OpenGL & glGetError() & GLenum & GL\_INVALID\_VALUE, GL\_OUT\_OF\_MEMORY\\
  \hline
  Cg toolkit & cgGetError()  & CGerror & CG\_NO\_ERROR, CG\_COMPILER\_ERROR\\
  \hline
  .NET4 System\-.Windows\-.Forms & GetError(Control) & String & System\-::Argument\-Exception, System::IO\-::Directory\-NotFound\-Exception\\
  \hline
\end{tabular}

V případě knihovny DCMTK je situace podobná jako u výše zmíněných knihoven. Po uvedeném volání konstruktoru na řádku 2 kódu \ref{DicomImageClass} můžeme zjistit stav jak vytvoření objektu dopadlo:

\begin{lstlisting}[label={xxx}]
iDicomImage = new DicomImage (fileName);

EI_Status imageStatus;
imageStatus = iDicomImage->getStatus();
\end{lstlisting}

Výčtový typ \clist{EI\_Status} může nabývat hodnot: EIS\_Normal, EIS\_InvalidValue, atd.

Pomocí funkce \clist{DicomImage::getStatus()} bylo tedy možné zjistit, že v debug verzi načítání snímku proběhně s chybou, a to konkrétně:

 \clist{EIS\_MissingAttribute}

Po zkoumání, co tato chyba znamená, vyplynulo, že načítaný DICOM soubor nemá svou hlavičku ve vyhovujícím stavu. Lze se dohledat, že nástroj: \clist{gdcmconv} dokáže nevyhovující hlavičku DICOM souboru převést do univerzálního tvaru, který bude akceptovat i knihovna DCMTK.

\subsection{Integrace externího nástroje do programu}
Zmiňovaný nástroj \clist{gdcmconv} dokáže opravit nekompatibilní hlavičku DICOM souboru. Otázkou bylo, jak tento nástroj do našeho programu integrovat. Nabízely se dvě odlišné varianty. První variantou by bylo:

\begin{itemize}
\item
Přeložit zdrojový kód programu \clist{gdcmconv} tak, že si vytvoříme C++ knihovnu. Knihovnu pak připojíme do programu DicomPresenter tak, že funkce programu \clist{gdcmconv} budeme moct používat uvnitř našeho programu.
\end{itemize}

Problém však je v tom, že program DicomPresenter už tak závisí na šesti různých knihovnách a tím se značně komplikuje jeho překlad. Při překladu aplikace by se pak programátor musel vázat na další knihovnu a musel by vždy obstarat zdrojový kód knihovny a přeložit si ji do \clist{.lib} souborů. V případě, že by zdrojový kód nenašel, nebo by se mu knihovnu programu \clist{gdcmconv} nepodařilo přeložit, nemohl by pak přeložit ani program DicomPresenter. Více o problematice překladu programu závislého na několika externích knihovnách v části \ref{sec:preklad} (str. \pageref{sec:preklad}). Proto tedy druhá varianta:

\begin{itemize}
\item
Obstarat program \clist{gdcmconv} jakožto spustitelný soubor. \clist{gdcmconv} pak spouštět v C++ jako externí program funkcí \clist{system(...)}.
\end{itemize}

Tato možnost se jevila jako vhodnější. Pro překlad DicomPresenteru nebode potřeba zdrojový kód programu \clist{gdcmconv}, bude stačit hotový program zkopírovat do adresáře DicomPresenteru. Pokud programátor nebude schopný program obstarat, DicomPresenter poběží i bez něj za cenu, že nebude možné otevřít určité DICOM soubory.

Po těchto úvahách byla implementována druhá možnost řešení. Kód funkce \clist{LoadDicomImage} (viz. kód \ref{DicomImageClass}) byl upraven následujícím způsobem:

\begin{lstlisting}[label={xxx}]
bool CDicomFrames::LoadDicomImage(char *fileName, bool isFirst, int framesCount ) {
	iDicomImage = new DicomImage (fileName);
	EI_Status imageStatus = iDicomImage->getStatus();

	if (imageStatus==EIS_MissingAttribute){			
		QString strCommand("gdcmconv --raw --force ");
		strCommand.append(fileName);
		strCommand.append(" temp.dcm");
		int i = system(strCommand.toAscii().data());
		if(i==0){
			iDicomImage=NULL;
			iDicomImage = new DicomImage ("temp.dcm");
			remove("temp.dcm");
		}else{
			QMessageBox msgBox;
			msgBox.setText("DICOM file is missing some header attribute.");
			msgBox.exec();
		}		
	}
 ...
}
\end{lstlisting}

Na řádku 3 si zjistíme, jak dopadlo načítání souboru. Pokud došlo k chybě (řádek 5), pak spouštíme externí program \clist{gdcmconv} s danými parametry (na řádcích 6-9) je sestavován celý příkaz. Příkaz:

\clist{gdcmconv --raw --force <nazevsouboru> temp.dcm}

nechá převést nevyhovující DICOM soubor na nový bezproblémový DICOM soubor. Pokud vytvoření nového souboru proběhně bez problému (řádek 10), pak je namísto původního souboru načten soubor nový. V opačném případě se uživateli objeví chybová hláška.

Po uvedené úpravě v kódu se výrazně zvýšil počet snímků, které DicomPresenter dokáže otevřít. Kód programu se nijak výrazně nezkomplikoval, ani by nový kus kódu neměl způsobit nějakou další chybu.



