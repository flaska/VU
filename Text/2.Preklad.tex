\newpage
\chapter{Konfigurace pro překlad}
\label{sec:preklad}
Program Dicom-Presenter využívá celkem šest externích knihoven. Překlad programu se tak stává komplikovaný. Při překladu programu je potřeba obstarat všechny knihovny ve zkompilované podobě. Některé knihovny lze stáhnout jako přeložené, některé si programátor musí přeložit sám.

Při sestavování DicomPresenteru pro použití v IKEM se objevila chybová hláška:

\clist{error LNK2005: <funkce> already defined in msvcprt.lib(MSVCP90.dll)}

Na MSDN\footnote{Microsoft Developer Network\cite{msdn}} lze dohledat, že přesně tato chybová hláška je zapříčiněna kombinováním různých, navzájem nekompatibilních, tzv. konfigurací pro překlad.

Konfigurace se v tomto případě odlišují podle toho, zda je kód překládán jako statická nebo dynamická knihovna, zda je překládán s podporou více vláken a zda je přeložen v debug nebo release verzi. Šest různých konfigurací, které je kvůli této chybové hlášce potřeba rozlišovat, je možno vidět v tabulkách \ref{table:konfigurace1}, \ref{table:konfigurace2}.

Důvodem, proč je potřeba uvedené konfigurace rozlišovat je to, že v každé z konfigurací je k programu přikládána jiná verze standardní C++ knihovny (C++ runtime). Různé verze standardní knihovny popisují stále stejné funkce. Přeložíme-li dvě části programu s různými standardními knihovnami, nemůžeme pak tyto části programu spojit do spustitelného souboru. Neboť v každé části programu jsou stejné funkce definovány jinak. Kompilátor hlásí chybu: already defined.

Standardní knihovna C++ obsahuje základní funkce, které se používají téměř ve všech programech: funkce pro práci s řetězci, funkce pro alokaci paměti, funkce pro textový výstup aplikace. V následujícím textu jsou rozebrány vlastnosti konfigurací.

\begin{table}[ht]
	\caption{Release onfigurace překladu, které je potřeba rozlišit kvůli standardní knihovně.}
  \label{table:konfigurace1}
	\centering
\begin{tabular}{| l|| p{5cm} | p{5.5cm}  |}
  \hline                       
  Release & Single-Thread & Multi-Thread \\
  \hline
  \hline                     
  Static Link & Single-Thread, Static Link, Release & Multi-Thread, Static Link, Release\\
  \hline
  Dynamic Link & --- & Multi-Thread, Dynamic Link, Release\\
  \hline  
\end{tabular}

\end{table}

\begin{table}[ht]
	\caption{Debug konfigurace překladu, které je potřeba rozlišit kvůli standardní knihovně.}
  \label{table:konfigurace2}
	\centering
\begin{tabular}{| l|| p{5cm} | p{5.5cm}  |}
  \hline                       
  Debug & Single-Thread & Multi-Thread \\
  \hline
  \hline                     
  Static Link & Single-Thread, Static Link, Debug & Multi-Thread, Static Link, Debug\\
  \hline
  Dynamic Link & --- & Multi-Thread, Dynamic Link, Debug\\
  \hline  
\end{tabular}

\end{table}

\section{Statické vs. dynamické knihovny}
Při překladu dílčí části kódu si musíme vybrat, zda kód chceme kompilovat do statické nebo dynamické knihovny. Rozdíl je v přenositelnosti programu na jiné počítače a ve velikosti spustitelného souboru. Překládáme-li součásti programu jako dynamické knihovny, musíme pak přeložené soubory distribuovat spolu se spustitelným souborem. Pokdu přeložíme všechny části programu jako statické knihovny, linker po dokončení překladu spojí všechny statické knihovny do jednoho spustitelného souboru.

Výhodou statického kompilování je snadná přenositelnost výsledného programu. Všechny používané součásti programu jsou linkovány uvnitř spustitelného souboru a nemůže se stát, že by na cílovém počítači nějaká knihovna scházela. Výhodou dynamického linkování je menší velikost spustitelného souboru a dále to, že knihovna je v paměti RAM jen jednou pro více aplikací, které jí sdílejí. V praxi u rozsahlých aplikací se proto hojně používá dynamické linkování.

\section{Single-thread vs. Multi-thread}
Při překladu je dále třeba si vybrat, zda má být aplikace přeložena s podporou více vláken. Připojení runtime knihovny s vícevláknovou podporou pak umožní používání příkazů k práci s vlákny.

Vícevláknové programování se používá např. u aplikací s grafickým rozhraním, kde je potřeba provést nějakou časově náročnější operaci, vedle toho je ale potřeba neustále obsluhovat grafické rozhraní. Dobrým příkladem je emailový klient: po stisknutí tlačítka odeslat začne odesílání emailu, tj. komunikace se serverem, přenášení dat atd. Během těchto akcí by uživatel u jednovláknové aplikace musel čekat, až vše doběhne do konce. Rozdělíme-li odesílání mailu a chod GUI do dvou nezávislých vláken, může uživatel manipulovat s programem i během odesílání emailu.

Bohužel používání více vláken je věc úzce spjatá s operačním systémem, pro OS Windows se používají funkce z hlavičkového souboru windows.h. Proto musíme při psaní aplikace, jejíž kód má být přenositelný mezi více operačními systémy použít nějakou knihovnu, která práci s vlákny obstará.




\section{Realase vs. Debug}
\label{releasedebug}
Další z možností výběru je Debug vs. Release. Překlad se liší v tom, že při debug překladu jsou do kódu vloženy dodatečné informace, které slouží k ladění programu. V programu jsou tabulky symbolů pomocí kterých lze zjišťovat stav proměnných. Dále lze zjistit jakému řádku ve zdrojovém kódu odpovídá aktuálně vykonávaný příkaz. Program je kompilován bez optimalizací. Release verze neobsahuje uvedené informace a navíc je optimalizována. Jak je vidět, bez informací o proměnných, o pozici aktuálně vykonaného řádku a při přeskupení kódu není možné program ladit. Běh programu je ale díky optimalizacím rychlejší.


\section{Verze knihovny C++ Runtime}
V tabulce \ref{table:stdlib} je vidět, kterou verzi standardní knihovny používají zmíněné konfigurace. V každém programu lze použít pouze jednu verzi standardní knihovny. Z toho vyplývá, že lze v programu kombinovat součásti přeložené jako single-thread a multi-thread. Nelze v programu kombinovat součásti přeložené jako static-link a dynamic-link. Nelze kombinovat ani relase a debug verze. Totéž platí i o externích knihovnách. Použijeme-li k překladu naší aplikace zkompilované verze externích knihoven, musejí být zkompilovány ve stejné, nebo kompatibilní konfiguraci jako náš program. Všechny externí knihovny musejí být zkompilovány v navzájem kompatibilních konfiguracích.

\begin{table}[ht]
	\caption{Verze standardní knihovny C++ pro jednotlivé konfigurace.\cite{msdn}}
  \label{table:stdlib}
	\centering
\begin{tabular}{| p{3cm} | p{3cm} | p{7.5cm} | }
  \hline                       
  Název C++ Runtime souboru & Odpovídající .dll knihovna & Popis konfigurace\\
  \hline
  \hline                     
  libcmt.lib & ---  & Multi-Thread, Static Link nebo Single-Thread, Static Link\\
  \hline
  msvcrt.lib & msvcr90.dll  & Multi-Thread, Dynamic Link\\
  \hline
  libcmtd.lib & ---  & Multi-Thread, Static Link, Debug nebo Single-Thread, Static Link, Debug\\
  \hline
  msvcrtd.lib & msvcr90d.dll & Multi-Thread, Dynamic Link, Debug\\
  \hline
\end{tabular}
\end{table}


