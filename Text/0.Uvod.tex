\chapter*{Úvod}
\addcontentsline{toc}{chapter}{Úvod}
\vspace{-10mm}
Tento výzkumný úkol si klade za cíl pokračování ve vývoji prohlížeče snímků pro Institut Klinické a Experimentální Medicíny. Jedná se o prohlížeč snímků pořízených na přístroji magnetické resonance. Vývoj programu byl započat v práci \cite{neskudla}, která si kladla za cíl objevit postupy pro programování prohlížeče. Při vývoji programu byl kladen vysoký důraz na jeho plynulý provoz. Pro vývoj byl vybrán jazyk C++ a pro zobrazování dat byla použita knihovna OpenGL. Zmiňovaná práce hledala zejména odpověď na otázku, zda je možné využít možností moderních grafických karet pro provoz prohlížeče. V dalším textu se budeme držet označeni pro program: DicomPresenter.

Tato práce řešila dvě úlohy. V první řadě je to pokračování ve vývoji DicomPresenteru. Při pokusu o zprovoznění testovací verze DicomPresenteru na dalších počítačích se začaly objevovat různé chyby v kódu. Popis jejich příčin a jejich ostranění je v práci popsáno. Odstranění všech chyb je nutná podmínka k tomu, aby program mohl být nasazen. Do programu byla dále přidána nová funkce pro zobrazování snímků.

Tento VÚ navazuje na práci \cite{neskudla} i v teoretické rovině, ale jde zcela odlišnou cestou. V tomto VÚ se snažíme odpovědět na otázku, jak výrazný bude pokles plynulosti běhu programu po odstranění OpenGL. Při zprovozňování DicomPresenteru na dalších počítačích opakovaně vznikal problém s nekompatibilitou grafické karty cílového počítače s rozšířeními OpenGL, které DicomPresenter používá. Z tohoto důvodu se zajímáme o to, jaký dopad na program by mělo odstranění OpenGL: Na základě toho rozhodneme, zda je cena za odstranění OpenGL příliš vysoká a proto bude knihovna v programu zachována, nebo naopak bude lepší jí odstranit.


